\documentclass{article}
\usepackage{amsmath}
%\usepackage{ulem}%uline命令宏包
\begin{document}

\section{sty1.1}%连字
it's difficult to find\ldots.\\
it's difficult to f{}ind\ldots.


\section{sty2.1}%标点符号(引号,连字符和破折号,省略号,波浪号)
``Please press the`x' key.''\\

X-rated\\13--67\\yes---or no?\\

one,two,three,\ldots one hundred.\\

a\~{}z \qquad a$\sim$z


\section{sty3.1}%文字强调
An example of \underline{some long and underlined words.}
%An example of \unline{some long and underlined\\ words.}

Some \emph{emphasized} words, including double-emphasized words, are shown here.


\section{sty4.1}%断词
I think this is: supercalifragilisticexpialidocious.\\
I think this is: su\-per\-cal\-ifrag\-ilistic\-ex\-pi\-alidocious.


\section{sty5.1}%引用
\begin{table}[h]
  \centering
\begin{tabular}{ccccc}
table & center & height & weight & name\\
1 & wang & 52 & 170 & jian\\
2 & kuang & 62 & 172 & qian\\
3 & li & 43 & 173 & guan\\
4 & zhang & 78 & 174 & yuan\\
\end{tabular}
  \caption{my first table }\label{tab1}
\end{table}

\ref{tab1}
\pageref{tab1}


\section{sty6.1}%脚注
\begin{tabular}{l}
\hline
A reference to this subsection looks like\footnotemark\\
\hline
\end{tabular}

\footnotetext{this is my article}


\section{sty7.1}文本对齐方式
\begin{center}
  Centered text using a center environment.
\end{center}

\begin{flushleft}
  Left-aligned text using a flushleft environment.
\end{flushleft}

\begin{flushright}
  Right-aligned text using a flushright environment.
\end{flushright}

\centering
 Centered text using a center environment.\\
\raggedright
Left-aligned text using a flushleft environment.\\
\raggedleft
Right-aligned text paragraph.\\

\raggedright
\section{sty8.1}%引用环境
Francis Bacon says:
\begin{quote}
  Knowledge is power.
\end{quote}

\begin{quotation}
Beauty is truth’s smile when she beholds her own face in a perfect mirror.
\end{quotation}


\section{sty9.1}%代码环境
\begin{verbatim}
  #include <iostream>
int main()
{
std::cout << "Hello, world!"
<< std::endl;
return 0;
}
\end{verbatim}

\verb+(a || b)+


\end{document}
